% Template for a CDF graph
%
% Author: Zengwen Yuan
% Version: 1.0  2017-03-08 init version
% Note: to use package fontspec,
% use XeLaTeX to compile

\documentclass{standalone}
\usepackage{siunitx}
% \usepackage{tikz}
\usepackage{pgfplots}
\usepackage{pgfplotstable}
%\usepackage{verbatim}
\usepackage{tikz}
\usepackage{helvet,etoolbox}
%\usepackage{sansmath}
\usepackage[active,tightpage]{preview}
\PreviewEnvironment{tikzpicture}
\setlength\PreviewBorder{1pt}
\usetikzlibrary{patterns}
\usetikzlibrary{pgfplots.statistics}

\pgfplotsset{compat=newest}

%\usepackage{fontspec}
%\setmainfont[
%BoldFont={Arial Bold},
%ItalicFont={Arial Italic},
%BoldItalicFont={Arial Bold Italic}
%]{Arial}

%\AtBeginEnvironment{tikzpicture}{\sansmath}
%\AtEndEnvironment{tikzpicture}{\unsansmath}

% # user-study-sync-overhead-context-tx-v2.txt
%\pgfplotstableread{
%0   0
%1000   0.1
%5000   0.8
%10000   0.95
%20000   1
%}{\fakedata}

\makeatletter
\pgfplotsset{
my filter/.style args={every#1between#2and#3}{%
/pgfplots/x filter/.append code={%
\ifnum\coordindex<#2%
  % Nothing
\else% Did we pass #3?
  \ifnum\coordindex>#3%
    %Nothing
  \else% Ok filter is on, don't disturb \pgfmathresult for convenience
    \pgfmathsetmacro\temp{int(mod(\coordindex,#1))}%
    \ifnum0=\temp\relax% Are we on the nth point?
      % Yes do nothing let it pass
    \else% discard it
      \let\pgfmathresult\pgfutil@empty
    \fi%
  \fi%
\fi%
}}}
\makeatother



\pgfplotsset{
    compat=newest,
    legend image code/.code={
        \draw[mark repeat=2,mark phase=2]
        plot coordinates {
            (0cm,0cm)
            (0.3cm,0cm)        %% default is (0.3cm,0cm)
            (0.6cm,0cm)         %% default is (0.6cm,0cm)
        };%
    }
}

\makeatletter
\pgfplotsset{
	boxplot prepared from table/.code={
		\def\tikz@plot@handler{\pgfplotsplothandlerboxplotprepared}%
		\pgfplotsset{
			/pgfplots/boxplot prepared from table/.cd,
			#1,
		}
	},
	/pgfplots/boxplot prepared from table/.cd,
	table/.code={\pgfplotstablecopy{#1}\to\boxplot@datatable},
	row/.initial=0,
	make style readable from table/.style={
		#1/.code={
			\pgfplotstablegetelem{\pgfkeysvalueof{/pgfplots/boxplot prepared from table/row}}{##1}\of\boxplot@datatable
			\pgfplotsset{boxplot/#1/.expand once={\pgfplotsretval}}
		}
	},
	make style readable from table=lower whisker,
	make style readable from table=upper whisker,
	make style readable from table=lower quartile,
	make style readable from table=upper quartile,
	make style readable from table=median,
	make style readable from table=lower notch,
	make style readable from table=upper notch,
}
\makeatother

\pgfplotstableread{
	lw lq med  uq uw
-20.13	-15.88	-12.88	-11.41	-10.64
-22.91	-20.05	-15.66	-10.49	-10.4
-15.24	-13.85	-12.58	-11.36	-10.84
-16.24	-14.48	-12.73	-11.66	-11.26
-18.68	-15.83	-13.11	-11.48	-10.6
}\datatable

\pgfplotstableread{
	x y
1	-42.95
1	-42.95
1	-42.95
1	-42.95
1	-42.95
1	-42.95
1	-43
1	-43
1	-43
1	-43
1	-43
2	-23.45
2	-23.96
2	-23.96
2	-23.96
2	-23.96
2	-23.96
2	-23.96
3	-16.13
3	-16.88
3	-17.06
3	-16.92
3	-16.83
3	-16.58
3	-16.55
3	-16.57
3	-16.81
3	-17.25
3	-17.56
3	-17.66
4	-17.52
4	-17.13
4	-16.91
4	-16.59
4	-16.47
4	-16.59
4	-16.63
4	-16.84
4	-17.02
4	-17.07
4	-17.14
5	-18.94
5	-18.98
5	-19.49
5	-19.16
5	-18.95
5	-18.95
5	-19.13
5	-19.74
5	-19.77
5	-19.56
5	-20.07
5	-20.34
5	-20.09
5	-19.02
5	-18.8
}\outlierdatatable

% \definecolor{palette1}{RGB}{215,25,28}
% \definecolor{palette2}{RGB}{253,174,97}
% % \definecolor{palette3}{RGB}{255,255,191}
% \definecolor{palette3}{RGB}{208,28,139}
% \definecolor{palette4}{RGB}{184,225,134}
% % \definecolor{palette4}{RGB}{171,221,164}
% \definecolor{palette5}{RGB}{43,131,186}
% \definecolor{palette6}{RGB}{208,28,139}

% green -- blueish
% \definecolor{palette1}{RGB}{0,109,44}
% \definecolor{palette2}{RGB}{44,162,95}
% \definecolor{palette3}{RGB}{67,162,202}
% \definecolor{palette4}{RGB}{8,104,172}
% \definecolor{palette5}{RGB}{8,81,156}

%\definecolor{myred}{RGB}{202,0,32}
%\definecolor{myorange}{RGB}{244,165,130}
%\definecolor{myviolet}{RGB}{194,165,207}
%\definecolor{mycyan}{RGB}{146,197,222}
%\definecolor{myblue}{RGB}{5,113,176}
%\definecolor{mygreen}{RGB}{127,191,123}
%\definecolor{mytile}{RGB}{27,120,55}
%\definecolor{myblack}{RGB}{60,60,60}
%% \definecolor{palette3}{RGB}{247,247,247}

\definecolor{myred}{HTML}{C00000}
\definecolor{mypurple}{HTML}{7030A0}
\definecolor{myblue}{HTML}{0070C0}
\definecolor{myblue2}{HTML}{0431FF}
\definecolor{myblue0}{HTML}{0000FF}
\definecolor{mygreen}{HTML}{00B050}
\definecolor{mygray}{HTML}{A6A6A6}


% \documentclass{standalone}
\usepackage{pgfplots}
% Nice color sets, see see http://colorbrewer2.org/	
\usepgfplotslibrary{colorbrewer}
% initialize Set1-4 from colorbrewer (we're comparing 4 classes),
\pgfplotsset{compat = 1.15, cycle list/Set1-8} 
% Tikz is loaded automatically by pgfplots
\usetikzlibrary{pgfplots.statistics, pgfplots.colorbrewer} 
% provides \pgfplotstabletranspose
\usepackage{pgfplotstable}
\usepackage{filecontents}

\begin{document}

%\begin{tikzpicture}[font=\sffamily]
\begin{tikzpicture}
	
\begin{axis}[
    %ymode = log,
    boxplot/draw direction=y,
    ymax    = -5,
    ymin    = -30,
    xmax    = 5.5,
    xmin    = 0.5,
    width   =  5 cm,
    height  =  2.2 cm,
    scale only axis = true,
    xtick align = inside,
    % bar width   = 1pt,
    % enlarge x limits = auto,
    % xticklabel style = {
    %     font=\small,
    %     % at={(current axis.north west)},
    %     % rotate=-90,
    %     % xshift=-1ex,
    %     yshift = 2pt,
    % },
    tick label style={/pgf/number format/assume math mode=true},
    xtick = {1,2,3,4,5},
    xticklabels = {A1,A2,A3,A4,A5},
    %ytick = {1, 10, 20, 30, 40, 50, 60, 70, 80, 90, 100, 200, 300, 400},
    %yticklabels = {1, 10, , , , , , , , , 100, , , 400},
    ytick = {-25, -20, -15, -10}, 
    yticklabels = {-25, -20, -15, -10},  
    %ytick = {1, 10, 20, 30, 40, 50, 60, 70, 80, 90, 100, 200, 300, 400},
    %yticklabels =   {1, 10, , , , , , , , , 100, , , 400},
    %xticklabel style = {font=\small},
    %yticklabel style = {font=\small},
    boxplot/variable width,
    boxplot/box extend=0.8,
    ymajorgrids=true,
    % % xticklabel pos=right,
    % xtick pos = left,
    % ytick pos = left,
    % % yticklabel pos=right,
    % % enlargelimits=0.15,
    ylabel={RSRQ (dB)},
   % xlabel={D1 \hfill D2},
    xlabel style={
        %font=\small,
        % at={(current axis.north west)},
        % rotate=-90,
       % xshift= -10pt,
        yshift = 5pt,
    },
   yticklabel style = {
    %font=\normalsize,
   	yshift =  0 pt,
        xshift =  3 pt,
    },
       xticklabel style = {
    %font=\normalsize,
   	yshift =  2 pt,
        xshift =  0 pt,
    },
    ylabel style={
        %font=\small,
        % at={(current axis.north west)},
        % rotate=-90,
        %text width=2.8cm,
        align=center,
        xshift= -3pt,
        yshift = -8 pt,
    },
legend style={
	at      = {(1, 1)},
	font    = \normalsize,
	fill    = none,
	draw    = none,
	anchor  = north east,
	legend cell align   = left,
	legend columns      = 2,
},
%    xlabel style={
%        font=\large,
%        % at={(current axis.north west)},
%        % rotate=-90,
%        % xshift=-1ex,
%        xshift = 0 pt,
%        yshift = 3pt,
%    },
    % symbolic x coords={att, tmo, sprint, cmcc, fi},
    % xticklabels={AT\&T, T-Mobile, Sprint, China Mobile, Google-Fi},
    % xtick=data,
    % nodes near coords,
    % nodes near coords align={vertical},
    cycle list={{myred},{myblue}},
    ]
    
%		\foreach \n in {1,...,3} {
%			\addplot+[boxplot, fill, draw=black] table[y index=\n] {\datatransposed};
%		}    

%\pgfplotstablegetrowsof{\datatable}
%\pgfmathtruncatemacro\TotalRows{\pgfplotsretval-1}
%\pgfplotsinvokeforeach{0,...,\TotalRows}
%{
%	\addplot+[
%	boxplot prepared from table={
%		table=\datatable,
%		row=#1,
%		lower whisker=lw,
%		upper whisker=uw,
%		lower quartile=lq,
%		upper quartile=uq,
%		median=med
%	},
%	boxplot prepared={
%		every box/.style={thick,black,solid},
%		every whisker/.style={very thin,black,solid},
%		every median/.style={very thick,blue,solid},},
%	% to get a more useful legend
%	area legend
%	]
%	coordinates {};
%	% add legend entry 
%	%\pgfplotstablegetelem{#1}{name}\of\datatable
%	%\addlegendentryexpanded{\pgfplotsretval}
%}

\pgfplotstablegetrowsof{\datatable}
\pgfmathtruncatemacro\TotalRows{\pgfplotsretval-1}
\pgfplotsinvokeforeach{0,...,\TotalRows}
{
	\addplot+[
	boxplot prepared from table={
		table=\datatable,
		row=#1,
		lower whisker=lw,
		upper whisker=uw,
		lower quartile=lq,
		upper quartile=uq,
		median=med,
		%draw position = 1
	},
	boxplot prepared={
		draw position=#1+1,
		%		every box/.style={thick,black,solid},
		%		every whisker/.style={very thin,black,solid},
		%		every median/.style={very thick,blue,solid},},
	every box/.style={thick,black,solid},
	every whisker/.style={very thin,black,solid},
	every median/.style={very thick,myblue,solid},},
fill,fill opacity=0.05,
% to get a more useful legend
area legend
]
coordinates {};
add legend entry 
%\pgfplotstablegetelem{#1}{name}\of\datatable
%\addlegendentryexpanded{\pgfplotsretval}
}

%\node[text=black] at (axis cs:2,260) {\bf 1};
%\node[text=black] at (axis cs:6,260) {\bf 2};
%\node[text=black] at (axis cs:10,260) {\bf 3};
%\node[text=black] at (axis cs:14,260) {\bf 4};
%\node[text=black] at (axis cs:18,260) {\bf 5};
%\node[text=black] at (axis cs:22,260) {\bf 6};
%\node[text=black] at (axis cs:26,260) {\bf 7};
%\node[text=black] at (axis cs:30,260) {\bf 8};
%\node[text=black] at (axis cs:34,260) {\bf 9};
%\node[text=black] at (axis cs:38,260) {\bf 10};
%\node[text=black] at (axis cs:42,260) {\bf 11};
%\node[text=black] at (axis cs:46,260) {\bf 12};
%\node[text=black] at (axis cs:50,260) {\bf 13};

\addplot[only marks,draw=black,fill=white,mark=*,mark size=1pt,fill opacity=0.2] table[x=x,y=y] \outlierdatatable;

%\legend{\texttt{T}, \texttt{A}}

%\addplot+[mark=none, densely dashed,gray,line width=0.5pt] coordinates {(7.8,0) (7.8,210)};
%\node[] at (axis cs: 6,180) {{\bf D1}};
%\node[] at (axis cs: 9.5,180) {{\bf D2}};

%\draw [decorate, decoration={brace,amplitude=5pt,raise=1pt}] (1,150) -- (8,150) node [midway, anchor=north, yshift=10mm, outer sep=10pt,font=\small]{{\bf D1}};
%
%\draw [decorate, decoration={brace,amplitude=5pt,raise=1pt}] (10,150) -- (14,150) node [midway, anchor=north, yshift=10mm, outer sep=10pt,font=\small]{{\bf D2}};

%\filldraw [draw=myred, fill = myred,fill opacity=0.1, thick] (1.4,350) rectangle (2,700);
%\node[,right] at (axis cs: 2,500) {\normalsize RT};
%
%\filldraw [draw=myblue, fill = myblue,fill opacity=0.1, thick] (3.4,350) rectangle (4,700);
%\node[,right] at (axis cs: 4,500) {\normalsize Others};



%\node[,right] at (axis cs: 1.25, 400) {\small 300 Mbps};

% replot median in RED
%\addplot+[mark=none, very thick,solid,red] coordinates {(0.75,13.3) (1.25,13.3)};
%\addplot+[mark=none, very thick,solid,red] coordinates {(1.75,23.9) (2.25,23.9)};
%\addplot+[mark=none, very thick,solid,blue] coordinates {(2.75,165.1) (3.25,165.1)};


%\draw[gray,densely dashed,->,line width=0.2pt] (axis cs:1.25,13) -- (axis cs: 1.5,1.5);
%\node[,right] at (axis cs: 1.25,1.5) {\small 13.0};
%\draw[gray,densely dashed,->,line width=0.2pt] (axis cs:2.75,21.8) -- (axis cs: 3,6);
%\node[,right] at (axis cs: 2,6) {\small 21.8};
%\node[,right] at (axis cs: 1,300) {\small 164.5};

%\addplot+[mark=none, densely dashed,red,line width=0.2pt] coordinates {(0,13.3) (1.25,13.3)};
%\addplot+[mark=none, densely dashed,red,line width=0.2pt] coordinates {(2,23.9) (2.75,23.9)};
%\addplot+[mark=none, densely dashed,blue,line width=0.2pt] coordinates {(2.25,165.1) (3.5,165.1)};
%
%\draw[red,densely dashed,line width=0.2pt] (axis cs:1.25,13.3) -- (axis cs: 1.5,3);
%\node[red,right] at (axis cs: 1,2) {\small 13.3};
%\draw[red,densely dashed,line width=0.2pt] (axis cs:2.75,23.9) -- (axis cs: 3,7);
%\node[red,right] at (axis cs: 2,6) {\small 23.9};
%\draw[blue,densely dashed,line width=0.2pt] (axis cs:2.25,165.1) -- (axis cs: 1.5,300);
%\node[blue,right] at (axis cs: 1, 340) {\small 165.1};

%13
%21.8
%164.5

%\addplot+ [onlymarks,draw=black,mark=*,mark size=0.5pt,mark options={fill=black,},
%]table[x=x,y=y] \outlierdatatable;
%coordinates { 
%	(1,0)
%	(1, 100)
%	(2, 100)
%	(2, 100)
%};


\end{axis}
\end{tikzpicture}

\end{document}

