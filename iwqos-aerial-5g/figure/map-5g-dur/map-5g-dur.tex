% Template for a CDF graph
%
% Author: Zengwen Yuan
% Version: 1.0  2017-03-08 init version
% Note: to use package fontspec,
% use XeLaTeX to compile

%%\documentclass{article}
\documentclass{standalone}
\usepackage{siunitx}
% \usepackage{tikz}
\usepackage{pgfplots}
\usepackage{pgfplotstable}
%\usepackage{verbatim}
\usepackage{tikz}
\usepackage{helvet,etoolbox}
%\usepackage{sansmath}
\usepackage[active,tightpage]{preview}
\PreviewEnvironment{tikzpicture}
\setlength\PreviewBorder{1pt}
\usetikzlibrary{patterns}

\newlength\figureheight
\newlength\figurewidth
\setlength\figureheight{11cm}
\setlength\figurewidth{10cm}

\pgfplotsset{compat=newest}

%\pgfplotsset{
%%/pgfplots/colormap={jet2}{rgb255(0cm)=(194,165,207) rgb255(2cm)=(202,0,32) 
%%rgb255(4cm)=(55,198,223) rgb255(6cm)=(5,113,176) rgb255(8cm)=(127,191,123)}
%%/pgfplots/colormap={jet2}{rgb255(0cm)=(202,0,32)
%%rgb255(4cm)=(244,165,130) rgb255(8cm)=(5,113,176)}
%%/pgfplots/colormap={jet2}{rgb255(0cm)=(202,0,32) rgb255(4cm)=(60,60,60) rgb255(8cm)=(5,113,176)}
%/pgfplots/colormap={jet2}{rgb255(0cm)=(200,0,0) rgb255(4cm)=(240,240,240) rgb255(8cm)=(0,100,200)}
%}

%\pgfplotsset{
%%/pgfplots/colormap={jet2}{rgb255(0cm)=(194,165,207) rgb255(2cm)=(202,0,32) 
%%rgb255(4cm)=(55,198,223) rgb255(6cm)=(5,113,176) rgb255(8cm)=(127,191,123)}
%%/pgfplots/colormap={jet2}{rgb255(0cm)=(202,0,32)
%%rgb255(4cm)=(244,165,130) rgb255(8cm)=(5,113,176)}
%%/pgfplots/colormap={jet2}{rgb255(0cm)=(202,0,32) rgb255(4cm)=(60,60,60) rgb255(8cm)=(5,113,176)}
%/pgfplots/colormap={jet2}{rgb255(0cm)=(60, 30, 160) rgb255(2.5cm)=(60,113,254) rgb255(4cm)=(13,186,194) rgb255(4.5cm)=(204,192,39) rgb255(5cm)=(250,250,20)}
%}

%\pgfplotsset{
%%/pgfplots/colormap={jet2}{rgb255(0cm)=(194,165,207) rgb255(2cm)=(202,0,32) 
%%rgb255(4cm)=(55,198,223) rgb255(6cm)=(5,113,176) rgb255(8cm)=(127,191,123)}
%%/pgfplots/colormap={jet2}{rgb255(0cm)=(202,0,32)
%%rgb255(4cm)=(244,165,130) rgb255(8cm)=(5,113,176)}
%%/pgfplots/colormap={jet2}{rgb255(0cm)=(202,0,32) rgb255(4cm)=(60,60,60) rgb255(8cm)=(5,113,176)}
%/pgfplots/colormap={jet2}{rgb255(0cm)=(200,0,0) rgb255(2.5cm)=(204,192,39) rgb255(4cm)=(13,186,194) rgb255(4.5cm)=(60,113,254) rgb255(5cm)=(0,100,200)}
%}

\pgfplotsset{
%/pgfplots/colormap={jet2}{rgb255(0cm)=(194,165,207) rgb255(2cm)=(202,0,32) 
%rgb255(4cm)=(55,198,223) rgb255(6cm)=(5,113,176) rgb255(8cm)=(127,191,123)}
%/pgfplots/colormap={jet2}{rgb255(0cm)=(202,0,32)
%rgb255(4cm)=(244,165,130) rgb255(8cm)=(5,113,176)}
%/pgfplots/colormap={jet2}{rgb255(0cm)=(202,0,32) rgb255(4cm)=(60,60,60) rgb255(8cm)=(5,113,176)}
/pgfplots/colormap={jet2}{rgb255(0cm)=(200,0,0) rgb255(2.5cm)=(200,200,0) rgb255(4cm)=(100,200,100) rgb255(4.5cm)=(0,200,200) rgb255(5cm)=(0,100,200)}
}

%\pgfplotsset{
%    % this transformation ensures that every input argument is
%    % transformed from -0.2 : 0.5 -> -0.5,0.5 
%    % and every tick label is transformed back:
%    nonlinear colormap trafo/.code 2 args={
%        \def\nonlinearscalefactor{((#2)/(#1))}%
%        \pgfkeysalso{%
%            y coord trafo/.code={%
%                \pgfmathparse{##1 < 0 ? -1*##1*\nonlinearscalefactor : ##1}%
%            },
%            y coord inv trafo/.code={%
%                \pgfmathparse{##1 < 0 ? -1*##1/\nonlinearscalefactor : ##1}%
%            },
%        }%
%    },
%    nonlinear colormap around 0/.code 2 args={
%        \def\nonlinearscalefactor{((#2)/(#1))}%
%        \pgfkeysalso{
%            colorbar style={
%                nonlinear colormap trafo={#1}{#2},
%                %
%                % OVERRIDE this here. The value is *only* used to
%                % generate a nice axis, it does not affect the data.
%                % Note that these values will be mapped through the
%                % colormap trafo as defined above.
%                point meta min={#1},
%                point meta max={#2},
%            },
%            %
%            % this here is how point meta is computed for the plot.
%            % It means that a point meta of -0.2 will actually become -0.5
%            % Thus, the *real* point meta min is -0.5... but we
%            % override it above.
%            point meta={y < 0 ? -y*\nonlinearscalefactor : y},
%        }%
%    },
%}

\makeatletter
\pgfplotsset{
	my filter/.style args={every#1between#2and#3}{%
		/pgfplots/x filter/.append code={%
			\ifnum\coordindex<#2%
			% Nothing
			\else% Did we pass #3?
			\ifnum\coordindex>#3%
			%Nothing
			\else% Ok filter is on, don't disturb \pgfmathresult for convenience
			\pgfmathsetmacro\temp{int(mod(\coordindex,#1))}%
			\ifnum0=\temp\relax% Are we on the nth point?
			% Yes do nothing let it pass
			\else% discard it
			\let\pgfmathresult\pgfutil@empty
			\fi%
			\fi%
			\fi%
}}}
\makeatother

\pgfplotsset{
	compat=newest,
	legend image code/.code={
		\draw[mark repeat=2,mark phase=2]
		plot coordinates {
			(0cm,0cm)
			(0.3cm,0cm)        %% default is (0.3cm,0cm)
			(0.6cm,0cm)         %% default is (0.6cm,0cm)
		};%
	}
}

\definecolor{myred}{RGB}{200,0,0}
\definecolor{myorange}{RGB}{250,150,100}
\definecolor{myviolet}{RGB}{194,165,207}
\definecolor{mycyan}{RGB}{146,197,222}
\definecolor{myblue}{RGB}{0,100,200}
\definecolor{mygreen}{RGB}{100,200,100}
\definecolor{mytile}{RGB}{27,120,55}
\definecolor{myblack}{RGB}{60,60,60}

%\definecolor{myblue}{HTML}{0070C0}
%\definecolor{myred}{HTML}{C00000}
%\definecolor{mypurple}{HTML}{7030A0}
%\definecolor{mygreen}{HTML}{00B050}

\pgfdeclareplotmark{mygridmark}
{%
%	\path[fill=mapped color] (-\pgfplotmarksize,-\pgfplotmarksize) rectangle (\pgfplotmarksize,1.3023*\pgfplotmarksize);
\path[fill=mapped color, fill opacity = 0.8, line width=0.2pt, color=mapped color] (-\pgfplotmarksize,-\pgfplotmarksize) rectangle (\pgfplotmarksize,20*\pgfplotmarksize);
}


\begin{document}
	
	\begin{tikzpicture}
		\begin{axis}[
		%%restrict x to domain=-86.168:-86.145,
		%%ymax    = 39.781, % 2200m
		%%ymin    = 39.761,
		xmin = 40.4606, % 2070m
		xmax = 40.4686,
		width   = 4.5cm,
		height  = 3.5cm,
		scale only axis = true,
		xtick align = inside,
		%axis line style={draw=mygray},
		%axis line style={draw=black},
		axis line style={draw=none},
		xtick style={draw=none},
			ytick={0, 30, 60, 90, 120},
			yticklabels={0m, 30m, 60m, 90m, 120m},
			%yticklabels={0m, 30m, 60m, 90m, 120m},
			xtick={40.4606, 40.46235,40.4641,40.46585,40.4676},
			xticklabels={0, 200m, 400m, 600m, 800m},
			%ylabel = {Altitude (m)},
		xlabel={Location grids along R1},
		xlabel style={
                        %font=\normalsize,
                        %xshift=-10pt,
                        yshift =5pt,
                    },
		ylabel = {},
		    ylabel style={
                        font=\normalsize,
                        %xshift=-10pt,
                        yshift =-10pt,
                    },
		    yticklabel style={
                        	font=\normalsize,
                        	align=right,
        		        	rotate=-90,
        			xshift=  -2mm,
        			yshift = 2 pt,
                    },
                    xticklabel style={
        			%font=\small,
        			align=center,
        			xshift= 0pt,
        			yshift = 10pt,
    		},
		%%ticks=none,
		colorbar,
		%nonlinear colormap around 0={0.5}{1},
		colormap/jet2,
		%colormap={whiteblue}{color=(white) color=(mygray) color=(120,120,250) color=(blue)},
		%colormap={whiteblue}{color=(white) color=(mygray2) color=(blue)},
           	% colormap={gb}{color=(240, 240, 240) color=(white) color=(green)},
		%%colormap={gray2black}{
			%%rgb255=(250, 250, 250)
			%%rgb255=(0,0,255)
		%%},
%	colormap={wbb}{
%		color(0)=(white)
%		color(900)=(blue)
%		color(1000)=(rgb:red,11;green,11;blue,69)
%	},
%	colormap/hot2,
		%colormap/gray2black,
		%colormap/whiteblue,
		%%colorbar horizontal,
		colorbar style={
			ytick={0.5, 0.7, 0.9, 0.95, 1},
			yticklabels={0.5,0.7,0.9,,1.0},
			%yticklabels={50\%,75\%,90\%,95\%,100\%},
			%ytick={0.5,0.75,1,1.25,1.5,1.75},
			%yticklabel=$10^{\pgfmathprintnumber{\tick}}$},
			%yticklabel={$10^{\pgfmathprintnumber{\tick}}$},
			%yticklabel={$10^{\pgfmathprintnumber{\tick}}$},
			%yticklabels={50,75,90,95,99,100},
			%%xticklabels={},
			%xticklabels={},
			%%xlabel={Duration Ratio},
			%				yticklabel style={text width=0.1cm,}
			%at={(0.05,0.18)},
			%%SSat={(0.05,0.9)},
			%%ymin=0,
			%%ymax=1,
			%anchor=north,
			%%height=0 cm,
			%%height=0.1cm,
			height=3.2cm,
			xshift= -2mm,
			yshift= 1mm,
			width=0.2cm,
			fill opacity=0.8, 
			tick label style={opacity=1,xshift= -3pt}
		}
	]
		
			\addplot[
			only marks,
			mark=mygridmark,
			mark size=0.83pt,
			scatter,
			scatter src=explicit,
			point meta=explicit,
			point meta max = 1,
			point meta min = 0.5,
			]
			%		file {data.dat};
			table[
			y = grid_height,
			x = grid_lon,
			meta = nr_duration_ratio,
			col sep=comma,
			]
			{nr_duration_ratio_0.0001_r1.csv};
			%{A-R1_connectivity_ratio_pixel5_0405-1009_310410_0.0005_0_0_0.95_0.05_grid_passed_5gm.csv};
			%{A-R1_connectivity_ratio_pixel5_0405-0930_310410_0.0005_0_0_0.95_0.05_grid_passed_5gm.csv};
			%{A-R1_connectivity_ratio_pixel5_0407_0807_310410_0.0005_0_0_0.95_0.05_grid_passed_5gm.csv};
			% {A-R1_connectivity_ratio_pixel5_0407_0807_310410_0.0005_0_0_area_type_matched_10_10_5gm.csv};
% R1: 39.761	39.781	-86.168	-86.145
% 1.97km x 2.22km
			%\pgfplotsextra{
				%				%subregion2 R1B
%				\draw[line width=0.3mm, subregioncolor2,-,>=latex] (axis cs:-86.166,39.7725) to (axis cs:-86.166,39.7665);
%				\draw[line width=0.3mm, subregioncolor2,-,>=latex] (axis cs:-86.166,39.7665) to (axis cs:-86.161,39.7665);
%				\draw[line width=0.3mm, subregioncolor2,-,>=latex] (axis cs:-86.161,39.7665) to (axis cs:-86.161,39.761);
%				\draw[line width=0.3mm, subregioncolor2,-,>=latex] (axis cs:-86.161,39.761) to (axis cs:-86.154,39.761);
%				\draw[line width=0.3mm, subregioncolor2,-,>=latex] (axis cs:-86.154,39.761) to (axis cs:-86.154,39.7665);
%				\draw[line width=0.3mm, subregioncolor2,-,>=latex] (axis cs:-86.154,39.7665) to (axis cs:-86.149,39.7665);
%				\draw[line width=0.3mm, subregioncolor2,-,>=latex] (axis cs:-86.149,39.7665) to (axis cs:-86.149,39.7705);
%				\draw[line width=0.3mm, subregioncolor2,-,>=latex] (axis cs:-86.149,39.7705) to (axis cs:-86.151,39.7705);
%				\draw[line width=0.3mm, subregioncolor2,-,>=latex] (axis cs:-86.151,39.7705) to (axis cs:-86.151,39.7725);
%				\draw[line width=0.3mm, subregioncolor2,-,>=latex] (axis cs:-86.151,39.7725) to (axis cs:-86.15,39.7725) ;
%				\draw[line width=0.3mm, subregioncolor2,-,>=latex] (axis cs:-86.15,39.7725) to (axis cs:-86.15,39.775);
%				\draw[line width=0.3mm, subregioncolor2,-,>=latex] (axis cs:-86.15,39.775) to (axis cs:-86.157,39.775);
%				\draw[line width=0.3mm, subregioncolor2,-,>=latex] (axis cs:-86.157,39.775) to (axis cs:-86.157,39.7725);
%				\draw[line width=0.3mm, subregioncolor2,-,>=latex] (axis cs:-86.157,39.7725) to (axis cs:-86.166,39.7725);
%				%\node[text=subregioncolor2] at (axis cs:-86.164,39.768) {R1B};
% R1A				%\node[text=subregioncolor1] at (axis cs:-86.164,39.775) {R1A};
				%}
			
		\end{axis}
%	\pgfresetboundingbox
%	\path
%	(current axis.south west) -- ++(-0in,-0in)
%	rectangle (current axis.north east) -- ++(0in,0in);
	\end{tikzpicture}
	
\end{document}